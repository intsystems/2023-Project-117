\documentclass[12pt, twoside]{article} 
\newcommand{\hdir}{.}
\usepackage{./jmlda}
\usepackage[round]{natbib}
\begin{document}
\title
	[поиск и восстановление зависимостей во временных рядах]
	{поиск и восстановление зависимостей во временных рядах}
\author
	[I. M. Latypov]
	{I. M. Latypov, E. Vladimirov, V. V. Strizhov}
\email
	{latypov.im@phystech.edu}
\organization	
	{MIPT}	
\abstract
	{
		При прогнозировании временных рядов, зависящих от других временных рядов, требуется решить задачу выявления связей между ними.
		Предполагается, что добавление связанных временных рядов в прогностическую модель повысит качество прогноза. В данной работе для
		обнаружения зависимостей между временными рядами предлагается использовать метод .... Для предсказания зависимости между временными 
		рядами предлагается использовать метод Neural CDE
		}
\bigskip
\noindent

\maketitle
\textbf{Ключевые слова}: \emph{Neural CDE, CCM, временные ряды}

\section{Введение}
	Работа посвящена задаче поиска причинно-следственных связей между временными рядами.
	Эта задача актуальна, поскольку часто на практике приходится работать с многомерными временными рядами \citep{dataset_mlru}, и учет зависимостей между координатами может улучшить качество предсказаний. 
		
	Существует множество методов для обнаружения связей между временными рядами. Среди них Тест Гренжера и метод сходящегося перекрестного отображения
	(convergent cross mapping, CCM) и другие, но у них есть существенные недостатки. Предложенный нами метод "крутое название метода" призван устранить их.
	
\section{Математическая постановка}
	Обозначим $T = \{t_1, ... t_k\}$ - моменты наблюдений.
	
	P.S. нужно будет перейти к вероятностной формулировке
	
	$\mathbf{X} = [x_1, ... , x_k]^T$ - многомерный временной ряд, $x_i \in mathcal{R}^n$ - наблюдения $X$ в момент времени $t_i$.

		Цель заключается в том, чтобы найти функцию(критерий) $F: \mathcal{R}^{n \times k } \rightarrow \mathcal R$, по значениям которой можно делать вывод о зависимостях между координатами временного ряда $X$.
	
\section{Связанные работы (related works)}
	Метод сходящегося перекрестного отображения(CCM)\citep{CCM}:
	
		компоненты временного ряда отображаются в траектороное подпространство и проверяется непрерывность отображения одной траектории на другую. Критерием зависимости служит "степень непрерывности"(пока так, потом исправлю). У метода есть такие проблемы, как использование всего датасета, квадратичное от длины ряза время работы. Так же ислледование \citep{CCM_PA} выделяет другие недостатки.
	
	Кросс корелляция:
	
		Проверяется корреляция компонент сдвинутых по времени компонент временного ряда. Критерием служит максимальная полученная корреляция. Недостатки метода - использование всего датасета, квадратичное от длины ряза время работы. Так же хорошо известно, что корреляция не является достаточным условием для зависимости случайных величин.
	
	Тест Гренжера:
	
		Пусть $U, V$ - временные ряды. $U$ предсказывают с использованием $V$ и без использования. При существенном улучшении предсказаний делается вывод о зависимости рядов. Проблема метода заключается в том, что он не дает представлений о виде зависимости рядов. К тому же предсказания могли не улучшиться из-за неверной модели.
	
\section{эксперименты}
	
\section{Теоритическое обоснование}
	
\section{Заключение}

\bibliographystyle{plain}
\bibliography{references}
	
	
\end{document}