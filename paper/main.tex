\documentclass[12pt, twoside]{article} 
\newcommand{\hdir}{.}
\usepackage{./jmlda/jmlda}

\begin{document}
\title
	[поиск и восстановление зависимостей во временных рядах]
	{поиск и восстановление зависимостей во временных рядах}
\author
	[I. M. Latypov]
	{I. M. Latypov}
\email
	{latypov.im@phystech.edu}
\organization	
	{MIPT}	
\abstract
	{
		При прогнозировании временных рядов, зависящих от других процессов, требуется решить задачу выявления связанных пар рядов.
		Предполагается, что добавление этих рядов в модель может поспособствовать повышению качества прогноза. В данной работе для
		обнаружения связей между рядами предлагается использовать метод .... Для восстановления зависимости между рядами предлагается
		использовать метод Neural CDE
		}
\bigskip
\noindent

\maketitle
\textbf{Ключевые слова}: \emph{раз ; два; три }

\section{Введение}
	Работа посвящена задаче поиска причинно-следственных связей между временными рядами и методам восстановления этих зависимостей.
	Эта задача актуальна, поскольку часто на практике возникает несколько процессов, и учет зависимостей меджу этими ними может улучшить качество предсказаний. 
		
	Существует множество методов для обнаружения связей между рядами. Среди них Тест Гренжера и метод сходящегося перекрестного отображения
	(convergent cross mapping, CCM) и другие, но у них есть существенные недостатки: тест Гренжера не дает
	представления о виде зависимости рядов, а CCM требуется применять сразу ко всему доступному диапазону данных. К тому же алгоритм
	выполняется за $O(n^2)$($n$ - длина ряда) для многомерного случая. Предложенный нами метод "крутое название метода" призван устранить эти недостатки.
	
	Для восстановления зависимости между рядами часто используют Recurrental Neural Networks (RNN), но такие модели дискретизируют время
	что может повлиять на результат предсказания. Мы же предлагаем использовать Neural СDE, которые делают попытку учесть непрерывность
	по времени. К тому же мы получаем механизм для обучения на рядах, информация от которых приходит с разными промежутками времени, подробнее об этом дальше
    

\section{Теоритическое обоснование}
	
	
	
	
\end{document}