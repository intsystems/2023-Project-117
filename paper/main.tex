\documentclass[12pt, twoside]{article} 
\newcommand{\hdir}{.}
\usepackage{./jmlda}
\usepackage[round]{natbib}
\begin{document}
\title
	[поиск и восстановление зависимостей во временных рядах]
	{поиск и восстановление зависимостей во временных рядах}
\author
	[I. M. Latypov]
	{I. M. Latypov}
\email
	{latypov.im@phystech.edu}
\organization	
	{MIPT}	
\abstract
	{
		При прогнозировании временных рядов, зависящих от других временных рядов, требуется решить задачу выявления связей между ними.
		Предполагается, что добавление связанных временных рядов в прогностическую модель повысит качество прогноза. В данной работе для
		обнаружения зависимостей между временными рядами предлагается использовать метод .... Для предсказания зависимости между временными 
		рядами предлагается использовать метод Neural CDE
		}
\bigskip
\noindent

\maketitle
\textbf{Ключевые слова}: \emph{Neural CDE, CCM, временные ряды}

\section{Введение}
	Работа посвящена задаче поиска причинно-следственных связей между временными рядами и методам предсказания их эволюции.
	Эта задача актуальна, поскольку часто на практике приходится работать с многомерными временными рядами \citep{dataset_mlru}, и учет зависимостей между координатами может улучшить качество предсказаний. 
		
	Существует множество методов для обнаружения связей между рядами. Среди них Тест Гренжера и метод сходящегося перекрестного отображения
	(convergent cross mapping, CCM) и другие, но у них есть существенные недостатки: тест Гренжера не дает
	представления о виде зависимости рядов, а CCM требуется применять сразу ко всему доступному набору данных. К тому же алгоритм
	выполняется за $O(n^2)$($n$ - длина ряда) для многомерного случая. Предложенный нами метод "крутое название метода" призван устранить эти недостатки.
	
	Для предсказания временных рядов часто используют Recurrental Neural Networks (RNN), но такие модели дискретизируют время
	что может повлиять на результат предсказания. Мы же предлагаем использовать Neural СDE, которые учитывают непрерывность
	по времени. К тому же мы получаем механизм для обучения на рядах, информация от которых приходит с разными промежутками времени.
	
	
	Работа посвящена обнаружению причинно-следственных связей между разнородными временными рядами


\section{Теоритическое обоснование}
	
\section{Заключение}

\bibliographystyle{plain}
\bibliography{references}
	
	
\end{document}