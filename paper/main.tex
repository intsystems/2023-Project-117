\documentclass[12pt, twoside]{article} 
\newcommand{\hdir}{.}
\usepackage{./jmlda}
\usepackage[round]{natbib}
\usepackage{rotating}
\begin{document}
\title
	[поиск и восстановление зависимостей во временных рядах]
	{поиск и восстановление зависимостей во временных рядах}
\author
	[I. M. Latypov]
	{I. M. Latypov, E. Vladimirov, V. V. Strizhov}
\email
	{latypov.im@phystech.edu}
\organization	
	{MIPT}	
\abstract
	{
		При прогнозировании временных рядов, зависящих от других временных рядов, требуется решить задачу выявления связей между ними.
		Предполагается, что добавление связанных временных рядов в прогностическую модель повысит качество прогноза. В данной работе для
		обнаружения зависимостей между временными рядами предлагается совместить Convergent Cross Mapping с ODE-RNN. 
		}
\bigskip
\noindent

\maketitle
\textbf{Ключевые слова}: \emph{Neural CDE, CCM, временные ряды}

\section{Введение}
	Работа посвящена задаче поиска причинно-следственных связей между временными рядами.
	Эта задача актуальна, поскольку часто на практике приходится работать с многомерными временными рядами \citep{dataset_mlru}, и учет зависимостей между координатами может улучшить качество предсказаний. 
		
		Существует множество методов для обнаружения связей между временными рядами. Среди них Тест Гренжера и метод сходящегося перекрестного отображения
	(convergent cross mapping, CCM) и другие. Идея работы CCM основана не теореме Таккенса (ссылка). Метод отображает временные ряды в траекторные многообразия и рассматривает зависимость траекторий эволюций этих рядов. Наша идея заключается в улучшении процесса построения отображения в траекторное многообразие. Для этого обучается модель ODE-RNN на эмбеддингах CCM алгоритма и скрытые состояния RNN слоев используются как новые эмбеддинги, и на них уже применяется метод CCM. 
		
		Преимуществом нашего метода перед CCM к тому же является возможность работать с рядами с нерегулярными наблюдениями.
	
	
\section{Связанные работы (related works)}
		

\newpage
\begin{sidewaystable}
	\caption{сравнение методов}
		\begin{tabular}{|l|l|l|l|}
			\hline
			Метод            & краткое описание                                                                                                                                                                                                                                                                                      & достоинства                                                                                                 & недостатки                                                                                                                                                                                                                          \\ \hline
			Тест \\ Гренжера    & \begin{tabular}[c]{@{}l@{}}Пусть $U, V$ - временные ряды. \\  $U$ предсказывают с  использованием \\$V$ и без использования.При существенном \\  улучшении предсказаний делается вывод \\ \\ о зависимости рядов.\end{tabular}                                                                & Легко применять                                                                                             & \begin{tabular}[c]{@{}l@{}}не дает представлений о виде \\ зависимости рядов. \\ К тому же предсказания могут не \\ улучшиться из-за неверной модели.\end{tabular}                                                            \\ \hline
			Кросс \\ Корреляция & \begin{tabular}[c]{@{}l@{}}Проверяется корреляция  сдвинутых \\ по времени компонент временного ряда. \\ Критерием служит максимальная \\ полученная корреляция.\end{tabular}                                                                                                                & Легко применять                                                                                             & \begin{tabular}[c]{@{}l@{}}Приходится использовать весь датасет\\ + квадратичное от длины ряда \\ \\ время работы. Так же хорошо известно, \\ \\ что корреляция не является достаточным \\ \\ условием зависимости\end{tabular} \\ \hline
			CCM              & \begin{tabular}[c]{@{}l@{}}компоненты временного ряда отображаются \\ \\ в траектороное  подпространство и\\  проверяется возможность отображения \\ одной траектории на другую.\\ \\  Критерием зависимости служит\\ \\  "степень непрерывности"\\ \\ (пока так, потом исправлю).\end{tabular} & \begin{tabular}[c]{@{}l@{}}Легко применять.\\  Работает лучше \\ методов предложенных \\ выше.\end{tabular} & \begin{tabular}[c]{@{}l@{}}использование всего датасета, \\ квадратичное от длины ряда время работы.\\  Так же ислледование \textbackslash{}citep\{CCM\_PA\} \\ \\ выделяет другие недостатки.\end{tabular}                   \\ \hline
			CMM +\\ ODE-RNN    & обучаем сетку, применяем CCM                                                                                                                                                                                                                                                                          & \begin{tabular}[c]{@{}l@{}}Точнее может выявлять \\ зависимости между \\ временными рядами\end{tabular}     & \begin{tabular}[c]{@{}l@{}}Нужно обучать на достаточно \\ \\ большом куске данных\end{tabular}                                                                                                                                      \\ \hline
		\end{tabular}
\end{sidewaystable}
	
\newpage

	
\section{Математическая постановка}
Обозначим $T = \{t_1, ... t_k\}$ - моменты наблюдений.

P.S. нужно будет перейти к вероятностной формулировке

$\mathbf{X} = [x_1, ... , x_k]^T$ - многомерный временной ряд, $x_i \in mathcal{R}^n$ - наблюдения $X$ в момент времени $t_i$.

Цель заключается в том, чтобы найти функцию(критерий) $F: \mathcal{R}^{n \times k } \rightarrow \mathcal R$, по значениям которой можно делать вывод о зависимостях между координатами временного ряда $X$.
\section{эксперименты}
	Провели эксперименты, можно посмотреть на гите
	
\section{Теоритическое обоснование}
	
\section{Заключение}

\bibliographystyle{plain}
\bibliography{references}
	
	
\end{document}